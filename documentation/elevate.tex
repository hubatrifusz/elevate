\documentclass{report}
\usepackage{minted}
\usepackage{xcolor}
\usepackage{tcolorbox}
\usepackage{babel}
\usepackage{titlesec}
\tcbuselibrary{minted}

\renewcommand{\theFancyVerbLine}{\textcolor{gray!50}{\scriptsize\arabic{FancyVerbLine}}}
\renewcommand{\contentsname}{Tartalomjegyzék}

\titleformat{\subsubsection}
{\large\bfseries}{}{0em}{}

\newtcolorbox{codeblock}[2][]{
  colback=black!85!gray,
  colframe=black!50!gray, % Gradient effect
  colbacktitle=black!50!gray,
  coltitle=white,
  title={\ttfamily#2},
  fonttitle=\footnotesize\bfseries,
  arc=5pt,
  boxrule=1pt,
  toptitle=4pt,
  bottomtitle=2pt,
  left=25pt,
  right=5pt,
  top=2pt,
  bottom=2pt,
  %sharp corners=south,
  #1
}

% \begin{codeblock}{MyClass.cs}
%   \begin{minted}[
%     style=one-dark,
%     breaklines,
%     linenos,
%     firstline=1,
%     gobble=0
%     ]{csharp}
%   public class MyClass
%   {
%       public void MyMethod()
%       {
%         List<App> apps = new();

%         for (int i = 0; i < 10; i++)
%         {
%             apps.Add(new App());
%         }
%       }
%   }
%   \end{minted}
% \end{codeblock}

\begin{document}

\title{Elevate Dokumentáció}
\author{Somlói Dávid
        \and
        Trifusz Huba
        \and
        Verba Viktor}
\date{\today}

\maketitle

\tableofcontents

\section{A projektről}
A téma kiválasztásánál arra törekedtünk, hogy egy, a hétköznapi élet során alkalmazható szoftvert készítsünk. Több opció is felmerült, azonban végül egy szokásformáló felület mellett döntöttünk, amit Elevate-nek neveztünk el, az egészséges, felemelő életmód jegyében. Az Elevate ösztönzi a felhasználókat, hogy új, pozitív szokásokat vezessenek be, miközben hatékonyan követhetik saját fejlődésüket, emellett hozzájárul életminőségük javításához és a fenntart-ható fejlődéshez.
\subsection{Az Elevate célja}
A szoftver célja, hogy a kliens az általa kívánt szokásokat fejlessze, vagy újakat építsen be a napirendjébe. Például, ha a felhasználó a dohányzásról szeretne leszokni, akkor monitorozni tudja a fogyasztását és különféle jutalmakat kap, ha tartja a felállított célját. Nem csak a rossz szokások követését biztosítja az applikáció, pozitív célokat is ki lehet tűzni, mint “Napi 10 fekvőtámasz" vagy “Hetente kitakarítani”. Egy szokás tartásához elengedhetetlen, hogy a beállított gyakorisággal teljesítsük a kitűzött kihívásokat. Ennek megkönnyítése érdekében az Elevate egy naptárszerű nézetben jeleníti meg a teendőket és emlékeztet azok elvégzésére. 
\section{Weboldal}
\section{Mobil Applikáció}
\section{Adatbázis}
\section{Backend}

\subsection{Végpontok}

\vspace{0.5cm}
\subsubsection{Autentikáció}
\begin{description}

  \item [\textbf{POST /api/auth/register}] - Regisztráció
  
    \vspace{0.5cm}
    \textbf{Request body:}
    \begin{codeblock}{JSON}
      \begin{minted}[breaklines, linenos, style=one-dark]{json}
{
  "email": "string($email)",
  "password": "string",
  // Min. 12 karakter, legalább egy kisbetű, nagybetű, szám és speciális karakter
  "confirmPassword": "string",
  "firstName": "string",
  "lastName": "string"
}
      \end{minted}
    \end{codeblock}

    \vspace{0.5cm}
    \textbf{Válaszok:}
    \begin{itemize}
      \item \textbf{200 OK} - Sikeres regisztráció
      
        \textbf{Response body:}
        \begin{codeblock}{JSON}
          \begin{minted}[breaklines, linenos, style=one-dark]{json}
{
  "id": "string($uuid)",
  "createdAt": "string($date-time)",
  "email": "string($email)",
  "firstName": "string",
  "lastName": "string",
  "profilePictureBase64": "string",
  "longestStreak": Number
}
          \end{minted}
        \end{codeblock}

      \item \textbf{400 Bad Request} - Hibás vagy hiányos adatok
      %\item \textbf{409 Conflict} - A megadott e-mail cím vagy felhasználónév már foglalt
    \end{itemize}

  \item [\textbf{POST /api/auth/login}] - Bejelentkezés
  
    \vspace{0.5cm}
    \textbf{Request body:}
    \begin{codeblock}{JSON}
      \begin{minted}[breaklines, linenos, style=one-dark]{json}
{
  "email": "string($email)",
  "password": "string"
}
      \end{minted}
    \end{codeblock}

    \vspace{0.5cm}
    \textbf{Válaszok:}
    \begin{itemize}
      \item \textbf{200 OK} - Sikeres bejelentkezés
      
        \textbf{Response body:}
        \begin{codeblock}{JSON}
          \begin{minted}[breaklines, linenos, style=one-dark]{json}
{
  "userId": "string($uuid)",
  "token": "string"
}
          \end{minted}
        \end{codeblock}
      \item \textbf{403 Forbidden} - Helytelen e-mail cím vagy jelszó

        \textbf{Response body:}
        \begin{codeblock}{JSON}
          \begin{minted}[breaklines, linenos, style=one-dark]{json}
"Invalid login details"
          \end{minted}
        \end{codeblock}
    \end{itemize}

\end{description}

\vspace{0.5cm}
\subsubsection{Felhasználó}
\begin{description}
  \item[\textbf{GET /api/user}] - Felhasználók lekérdezése email alapján
  
    \vspace{0.5cm}
    Keresés email cím részlet alapján. A lekérdezés oldalanként maximum 20 felhasználót ad vissza.

    \vspace{0.5cm}
    \textbf{Paraméterek:}
    \begin{itemize}
      \item \textbf{email: string} - A felhasználó e-mail címe
      \item \textbf{pageNumber: int} - A lekérdezni kívánt oldal száma
      \item \textbf{pageSize: int} - A lekérdezni kívánt felhasználók száma
    \end{itemize}

    \vspace{0.5cm}
    \textbf{Válaszok:}
    \begin{itemize}
      \item \textbf{200 OK} - Sikeres lekérdezés

        \textbf{Response body:}
        \begin{codeblock}{JSON}
          \begin{minted}[breaklines, linenos, style=one-dark]{json}
{
  "id": "string($uuid)",
  "createdAt": "string($date-time)",
  "email": "string($email)",
  "firstName": "string",
  "lastName": "string",
  "profilePictureBase64": "string",
  "longestStreak": Number
}
          \end{minted}
        \end{codeblock}

      \item \textbf{400 Bad Request} - Hibás vagy hiányos adatok
      \item \textbf{404 Not Found} - Nincs találat

      \textbf{Response body:}
      \begin{codeblock}{JSON}
        \begin{minted}[breaklines, linenos, style=one-dark]{json}
"No users found."
        \end{minted}
      \end{codeblock}
    \end{itemize}

  \item[\textbf{GET /api/user/\{id\}}] - Felhasználó lekérdezése azonosító alapján

    \vspace{0.5cm}
    \textbf{Paraméterek:}
    \begin{itemize}
      \item \textbf{id: string(\$uuid)} - A felhasználó azonosítója
    \end{itemize}

    \vspace{0.5cm}
    \textbf{Válaszok:}
    \begin{itemize}
      \item \textbf{200 OK} - Sikeres lekérdezés

        \textbf{Response body:}
        \begin{codeblock}{JSON}
          \begin{minted}[breaklines, linenos, style=one-dark]{json}
{
  "id": "string($uuid)",
  "createdAt": "string($date-time)",
  "email": "string($email)",
  "firstName": "string",
  "lastName": "string",
  "profilePictureBase64": "string",
  "longestStreak": Number
}
          \end{minted}
        \end{codeblock}
      \item \textbf{404 Not Found} - Nincs találat

        \textbf{Response body:}
        \begin{codeblock}{JSON}
          \begin{minted}[breaklines, linenos, style=one-dark]{json}
"User not found."
          \end{minted}
        \end{codeblock}
    \end{itemize}

    \item[\textbf{PATCH /api/user/\{id\}}] - Felhasználó módosítása azonosító alapján

    \vspace{0.5cm}
    \textbf{Paraméterek:}
    \begin{itemize}
      \item \textbf{id: string(\$uuid)} - A felhasználó azonosítója
    \end{itemize}

    \vspace{0.5cm}
    \textbf{Request body:}
    \begin{codeblock}{JSON}
      \begin{minted}[breaklines, linenos, style=one-dark]{json}
{
  "firstName": "string",
  "lastName": "string",
  "profilePictureBase64": "string"
}
      \end{minted}
    \end{codeblock}

    \vspace{0.5cm}
    \textbf{Válaszok:}
    \begin{itemize}
      \item \textbf{200 OK} - Sikeres módosítás

        \textbf{Response body:}
        \begin{codeblock}{JSON}
          \begin{minted}[breaklines, linenos, style=one-dark]{json}
{
  "id": "string($uuid)",
  "createdAt": "string($date-time)",
  "email": "string($email)",
  "firstName": "string",
  "lastName": "string",
  "profilePictureBase64": "string",
  "longestStreak": Number
}
          \end{minted}
        \end{codeblock}

      \item \textbf{404 Not Found} - Nincs találat

        \textbf{Response body:}
        \begin{codeblock}{JSON}
          \begin{minted}[breaklines, linenos, style=one-dark]{json}
"User not found."
          \end{minted}
        \end{codeblock}
    \end{itemize}

\end{description}

\vspace{0.5cm}
\subsubsection{Barátok}
\begin{description}
  \item[\textbf{GET /api/friendship/\{userId\}/friends}] - Barátok lekérdezése

    \vspace{0.5cm}
    A felhasználó barátainak lekérdezése a felhasználó azonosítója alapján.

    \vspace{0.5cm}
    \textbf{Paraméterek:}
    \begin{itemize}
      \item \textbf{userId: string(\$uuid)} - A felhasználó azonosítója
    \end{itemize}

    \vspace{0.5cm}
    \textbf{Válaszok:}
    \begin{itemize}
      \item \textbf{200 OK} - Sikeres lekérdezés

        \textbf{Response body:}
        \begin{codeblock}{JSON}
          \begin{minted}[breaklines, linenos, style=one-dark]{json}
{
  "id": "string($uuid)",
  "createdAt": "string($date-time)",
  "email": "string($email)",
  "firstName": "string",
  "lastName": "string",
  "profilePictureBase64": "string",
  "longestStreak": Number
}
          \end{minted}
        \end{codeblock}

      \item \textbf{404 Not Found} - Nincs találat
      
        \textbf{Response body:}
        \begin{codeblock}{JSON}
          \begin{minted}[breaklines, linenos, style=one-dark]{json}
"User has no friends."
          \end{minted}
        \end{codeblock}
    \end{itemize}

  \item[\textbf{POST /api/friendship}] - Barát hozzáadása
  
    \vspace{0.5cm}
    \textbf{Request body:}
    \begin{codeblock}{JSON}
      \begin{minted}[breaklines, linenos, style=one-dark]{json}
{
  "userId": "string($uuid)",
  "friendId": "string($uuid)"
}
      \end{minted}
    \end{codeblock}

    \vspace{0.5cm}
    \textbf{Válaszok:}
    \begin{itemize}
      \item \textbf{200 OK} - Sikeres hozzáadás

        \textbf{Response body:}
        \begin{codeblock}{JSON}
          \begin{minted}[breaklines, linenos, style=one-dark]{json}
{
  "id": "string($uuid)",
  "createdAt": "string($date-time)",
  "email": "string($email)",
  "firstName": "string",
  "lastName": "string",
  "profilePictureBase64": "string",
  "longestStreak": Number
}
          \end{minted}
        \end{codeblock}

      \item \textbf{400 Bad Request} - Hibás vagy hiányos adatok
      \item \textbf{404 Not Found} - Nincs találat

        \textbf{Response body:}
        \begin{codeblock}{JSON}
          \begin{minted}[breaklines, linenos, style=one-dark]{json}
"User not found."
          \end{minted}
        \end{codeblock}
    \end{itemize}

  \item[\textbf{DELETE /api/friendship}] - Barát törlése
  
    \vspace{0.5cm}
    \textbf{Paraméterek:}
    \begin{itemize}
      \item \textbf{userId: string(\$uuid)} - A felhasználó azonosítója
      \item \textbf{friendId: string(\$uuid)} - A barát azonosítója
    \end{itemize}

    \vspace{0.5cm}
    \textbf{Válaszok:}
    \begin{itemize}
      \item \textbf{200 OK} - Sikeres törlés

        \textbf{Response body:}
        \begin{codeblock}{JSON}
          \begin{minted}[breaklines, linenos, style=one-dark]{json}
{
  "id": "string($uuid)",
  "createdAt": "string($date-time)",
  "email": "string($email)",
  "firstName": "string",
  "lastName": "string",
  "profilePictureBase64": "string",
  "longestStreak": Number
}
          \end{minted}
        \end{codeblock}

      \item \textbf{400 Bad Request} - Hibás vagy hiányos adatok
      \item \textbf{404 Not Found} - Nincs találat

        \textbf{Response body:}
        \begin{codeblock}{JSON}
          \begin{minted}[breaklines, linenos, style=one-dark]{json}
"User not found."
          \end{minted}
        \end{codeblock}
    \end{itemize}
\end{description}

\section{Tesztelés}

\end{document}
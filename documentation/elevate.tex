\documentclass[12pt]{report}
\usepackage{minted}
\usepackage{xcolor}
\usepackage{tcolorbox}
\usepackage{babel}
\usepackage[T1]{fontenc}
\usepackage{textcomp}
\usepackage{titlesec}
\usepackage[hidelinks]{hyperref}
\usepackage{bookmark}
\tcbuselibrary{minted}

\renewcommand{\theFancyVerbLine}{\textcolor{gray!50}{\scriptsize\arabic{FancyVerbLine}}}
\renewcommand{\contentsname}{Tartalomjegyzék}

\definecolor{getColor}{HTML}{4CAF50}
\definecolor{postColor}{HTML}{2196F3}
\definecolor{patchColor}{HTML}{FF9800}
\definecolor{deleteColor}{HTML}{F44336}
\definecolor{putColor}{HTML}{9C27B0}

\newcommand{\httpGet}[1]{\colorbox{getColor}{\textbf{\textcolor{white}{GET}}}~#1}
\newcommand{\httpPost}[1]{\colorbox{postColor}{\textbf{\textcolor{white}{POST}}}~#1}
\newcommand{\httpPatch}[1]{\colorbox{patchColor}{\textbf{\textcolor{white}{PATCH}}}~#1}
\newcommand{\httpDelete}[1]{\colorbox{deleteColor}{\textbf{\textcolor{white}{DELETE}}}~#1}
\newcommand{\httpPut}[1]{\colorbox{putColor}{\textbf{\textcolor{white}{PUT}}}~#1}

\titleformat{\chapter}
  {\normalfont\Huge\bfseries}
  {\thechapter.}
  {0.5em}                      
  {} 

\newtcolorbox{codeblock}[2][]{
  colback=black!85!gray,
  colframe=black!50!gray, % Gradient effect
  colbacktitle=black!50!gray,
  coltitle=white,
  title={\ttfamily#2},
  fonttitle=\footnotesize\bfseries,
  arc=5pt,
  boxrule=1pt,
  toptitle=4pt,
  bottomtitle=2pt,
  left=25pt,
  right=5pt,
  top=2pt,
  bottom=2pt,
  % sharp corners=south,
  #1
}

% \begin{codeblock}{MyClass.cs}
%   \begin{minted}[
%     style=one-dark,
%     breaklines,
%     linenos,
%     firstline=1,
%     gobble=0
%     ]{csharp}
%   public class MyClass
%   {
%       public void MyMethod()
%       {
%         List<App> apps = new();

%         for (int i = 0; i < 10; i++)
%         {
%             apps.Add(new App());
%         }
%       }
%   }
%   \end{minted}
% \end{codeblock}

\begin{document}

\title{Elevate Dokumentáció}
\author{Somlói Dávid
        \and
        Trifusz Huba
        \and
        Verba Viktor}
\date{\today}

\maketitle

\setcounter{tocdepth}{3}
\tableofcontents

\chapter{A projektről}
\begin{sloppypar}
A téma kiválasztásánál arra törekedtünk, hogy egy, a hétköznapi élet során alkalmazható szoftvert készítsünk. Több opció is felmerült, azonban végül egy szokásformáló felület mellett döntöttünk, amit Elevate-nek neveztünk el, az egészséges, felemelő életmód jegyében. Az Elevate ösztönzi a felhasználó-kat, hogy új, pozitív szokásokat vezessenek be, miközben hatékonyan követhetik saját fejlődésüket, emellett hozzájárul életminőségük javításához és a fenntart-ható fejlődéshez.
\end{sloppypar}
\section{Az Elevate célja}
\begin{sloppypar}
A szoftver célja, hogy a kliens az általa kívánt szokásokat fejlessze, vagy újakat építsen be a napirendjébe. Például, ha a felhasználó a dohányzásról szeretne leszokni, akkor monitorozni tudja a fogyasztását és különféle jutal-makat kap, ha tartja a felállított célját. Nem csak a rossz szokások követését biztosítja az applikáció, pozitív célokat is ki lehet tűzni, mint “Napi 10 fekvőtámasz" vagy “Hetente kitakarítani”. Egy szokás tartásához elengedhet-etlen, hogy a beállított gyakorisággal teljesítsük a kitűzött kihívásokat. Ennek megkönnyítése érdekében az Elevate egy naptárszerű nézetben jeleníti meg a teendőket és emlékeztet azok elvégzésére. 
\end{sloppypar}
\chapter{Weboldal}
\chapter{Mobil Applikáció}
\chapter{Adatbázis}
\chapter{Backend}

\section{Végpontok}

\vspace{0.5cm}
\subsection{Autentikáció}
\begin{description}

  \item [\httpPost{/api/auth/register}] - Regisztráció
  
    \vspace{0.5cm}
    \textbf{Request body:}
    \begin{codeblock}{JSON}
      \begin{minted}[breaklines, linenos, style=one-dark]{json}
{
  "email": "string($email)",
  "password": "string",
  // Min. 12 karakter, legalább egy kisbetű, nagybetű, szám és speciális karakter
  "confirmPassword": "string",
  "firstName": "string",
  "lastName": "string"
}
      \end{minted}
    \end{codeblock}

    \vspace{0.5cm}
    \textbf{Válaszok:}
    \begin{itemize}
      \item \textbf{200 OK} - Sikeres regisztráció
      
        \textbf{Response body:}
        \begin{codeblock}{JSON}
          \begin{minted}[breaklines, linenos, style=one-dark]{json}
{
  "id": "string($uuid)",
  "createdAt": "string($date-time)",
  "email": "string($email)",
  "firstName": "string",
  "lastName": "string",
  "profilePictureBase64": "string",
  "longestStreak": Number
}
          \end{minted}
        \end{codeblock}

      \item \textbf{400 Bad Request} - Hibás vagy hiányos adatok
      
        \textbf{Response body:}
        \begin{codeblock}{JSON}
          \begin{minted}[breaklines, linenos, style=one-dark]{json}
"Password must contain at least 12 characters, an uppercase letter, a number and a symbol."
          \end{minted}
        \end{codeblock}

      \item \textbf{409 Conflict} - A megadott e-mail cím vagy felhasználónév már foglalt
        
        \textbf{Response body:}
        \begin{codeblock}{JSON}
          \begin{minted}[breaklines, linenos, style=one-dark]{json}
"Email is already taken."
          \end{minted}
        \end{codeblock}
    \end{itemize}

  \item [\httpPost{/api/auth/login}] - Bejelentkezés
  
    \vspace{0.5cm}
    \textbf{Request body:}
    \begin{codeblock}{JSON}
      \begin{minted}[breaklines, linenos, style=one-dark]{json}
{
  "email": "string($email)",
  "password": "string"
}
      \end{minted}
    \end{codeblock}

    \vspace{0.5cm}
    \textbf{Válaszok:}
    \begin{itemize}
      \item \textbf{200 OK} - Sikeres bejelentkezés
      
        \textbf{Response body:}
        \begin{codeblock}{JSON}
          \begin{minted}[breaklines, linenos, style=one-dark]{json}
{
  "userId": "string($uuid)",
  "token": "string"
}
          \end{minted}
        \end{codeblock}
      \item \textbf{403 Forbidden} - Helytelen e-mail cím vagy jelszó

        \textbf{Response body:}
        \begin{codeblock}{JSON}
          \begin{minted}[breaklines, linenos, style=one-dark]{json}
"Invalid login details"
          \end{minted}
        \end{codeblock}

      \item \textbf{400 Bad Request} - Hibás vagy hiányos adatok
    \end{itemize}

\end{description}

\vspace{0.5cm}

\subsection{Felhasználó}
\begin{description}
  \item[\httpGet{/api/user}] - Felhasználók lekérdezése email alapján
  
    \vspace{0.5cm}
    Keresés email cím részlet alapján. A lekérdezés oldalanként maximum 20 felhasználót ad vissza.

    \vspace{0.5cm}
    \textbf{Paraméterek:}
    \begin{itemize}
      \item \textbf{email: string} - A felhasználó e-mail címe
      \item \textbf{pageNumber: int} - A lekérdezni kívánt oldal száma
      \item \textbf{pageSize: int} - A lekérdezni kívánt felhasználók száma
    \end{itemize}

    \vspace{0.5cm}
    \textbf{Válaszok:}
    \begin{itemize}
      \item \textbf{200 OK} - Sikeres lekérdezés

        \textbf{Response body:}
        \begin{codeblock}{JSON}
          \begin{minted}[breaklines, linenos, style=one-dark]{json}
{
  "id": "string($uuid)",
  "createdAt": "string($date-time)",
  "email": "string($email)",
  "firstName": "string",
  "lastName": "string",
  "profilePictureBase64": "string",
  "longestStreak": Number
}
          \end{minted}
        \end{codeblock}

      \item \textbf{400 Bad Request} - Hibás vagy hiányos adatok
      \item \textbf{404 Not Found} - Nincs találat

      \textbf{Response body:}
      \begin{codeblock}{JSON}
        \begin{minted}[breaklines, linenos, style=one-dark]{json}
"No users found."
        \end{minted}
      \end{codeblock}
    \end{itemize}

  \item[\httpGet{/api/user/\{id\}}] - Felhasználó lekérdezése azonosító alapján

    \vspace{0.5cm}
    \textbf{Paraméterek:}
    \begin{itemize}
      \item \textbf{id: string(\$uuid)} - A felhasználó azonosítója
    \end{itemize}

    \vspace{0.5cm}
    \textbf{Válaszok:}
    \begin{itemize}
      \item \textbf{200 OK} - Sikeres lekérdezés

        \textbf{Response body:}
        \begin{codeblock}{JSON}
          \begin{minted}[breaklines, linenos, style=one-dark]{json}
{
  "id": "string($uuid)",
  "createdAt": "string($date-time)",
  "email": "string($email)",
  "firstName": "string",
  "lastName": "string",
  "profilePictureBase64": "string",
  "longestStreak": Number
}
          \end{minted}
        \end{codeblock}
      \item \textbf{404 Not Found} - Nincs találat

        \textbf{Response body:}
        \begin{codeblock}{JSON}
          \begin{minted}[breaklines, linenos, style=one-dark]{json}
"User not found."
          \end{minted}
        \end{codeblock}
    \end{itemize}

    \item[\httpPatch{/api/user/\{id\}}] - Felhasználó módosítása azonosító alapján

    \vspace{0.5cm}
    \textbf{Paraméterek:}
    \begin{itemize}
      \item \textbf{id: string(\$uuid)} - A felhasználó azonosítója
    \end{itemize}

    \vspace{0.5cm}
    \textbf{Request body:}
    \begin{codeblock}{JSON}
      \begin{minted}[breaklines, linenos, style=one-dark]{json}
{
  "firstName": "string",
  "lastName": "string",
  "profilePictureBase64": "string"
}
      \end{minted}
    \end{codeblock}

    \vspace{0.5cm}
    \textbf{Válaszok:}
    \begin{itemize}
      \item \textbf{200 OK} - Sikeres módosítás

        \textbf{Response body:}
        \begin{codeblock}{JSON}
          \begin{minted}[breaklines, linenos, style=one-dark]{json}
{
  "id": "string($uuid)",
  "createdAt": "string($date-time)",
  "email": "string($email)",
  "firstName": "string",
  "lastName": "string",
  "profilePictureBase64": "string",
  "longestStreak": Number
}
          \end{minted}
        \end{codeblock}

      \item \textbf{401 Unauthorized} - A felhasználó nincs bejelentkezve

      \item \textbf{404 Not Found} - Nincs találat

        \textbf{Response body:}
        \begin{codeblock}{JSON}
          \begin{minted}[breaklines, linenos, style=one-dark]{json}
"User not found."
          \end{minted}
        \end{codeblock}
    \end{itemize}

\end{description}

\subsection{Barátok}
\begin{description}
  \item[\httpGet{/api/friendship/\{userId\}/friends}] - Barátok lekérdezése

    \vspace{0.5cm}
    A felhasználó barátainak lekérdezése a felhasználó azonosítója alapján.

    \vspace{0.5cm}
    \textbf{Paraméterek:}
    \begin{itemize}
      \item \textbf{userId: string(\$uuid)} - A felhasználó azonosítója
    \end{itemize}

    \vspace{0.5cm}
    \textbf{Válaszok:}
    \begin{itemize}
      \item \textbf{200 OK} - Sikeres lekérdezés

        \textbf{Response body:}
        \begin{codeblock}{JSON}
          \begin{minted}[breaklines, linenos, style=one-dark]{json}
{
  "id": "string($uuid)",
  "createdAt": "string($date-time)",
  "email": "string($email)",
  "firstName": "string",
  "lastName": "string",
  "profilePictureBase64": "string",
  "longestStreak": Number
}
          \end{minted}
        \end{codeblock}

      \item \textbf{404 Not Found} - Nincs találat
      
        \textbf{Response body:}
        \begin{codeblock}{JSON}
          \begin{minted}[breaklines, linenos, style=one-dark]{json}
"User has no friends."
          \end{minted}
        \end{codeblock}
    \end{itemize}

  \item[\httpGet{/api/friendship/\{userId\}/friend-requests}]- Barátkérések ~\\
  \indent lekérdezése

    \vspace{0.5cm}
    \begin{sloppypar}
      A felhasználó barátkéréseinek lekérdezése a felhasználó azonosítója alapján.
    \end{sloppypar}

    \vspace{0.5cm}
    \textbf{Paraméterek:}
    \begin{itemize}
      \item \textbf{userId: string(\$uuid)} - A felhasználó azonosítója
    \end{itemize}

    \vspace{0.5cm}
    \textbf{Válaszok:}
    \begin{itemize}
      \item \textbf{200 OK} - Sikeres lekérdezés

        \textbf{Response body:}
        \begin{codeblock}{JSON}
          \begin{minted}[breaklines, linenos, style=one-dark]{json}
{
"id": "string($uuid)",
"createdAt": "string($date-time)",
"email": "string($email)",
"firstName": "string",
"lastName": "string",
"profilePictureBase64": "string",
"longestStreak": Number
}
          \end{minted}
        \end{codeblock}

      \item \textbf{404 Not Found} - Nincs találat

        \textbf{Response body:}
        \begin{codeblock}{JSON}
          \begin{minted}[breaklines, linenos, style=one-dark]{json}
"User has no friend requests."
          \end{minted}
        \end{codeblock}
    \end{itemize}

  \item[\httpGet{/api/friendship/\{userId\}/friend-requests-sent}] - Elküldött barátkérések lekérdezése
  
    \vspace{0.5cm}
    \begin{sloppypar}
      A felhasználó által elküldött barátkérések lekérdezése a felhasználó azonosítója alapján.
    \end{sloppypar}

    \vspace{0.5cm}
    \textbf{Paraméterek:}
    \begin{itemize}
      \item \textbf{userId: string(\$uuid)} - A felhasználó azonosítója
    \end{itemize}

    \vspace{0.5cm}
    \textbf{Válaszok:}
    \begin{itemize}
      \item \textbf{200 OK} - Sikeres lekérdezés

        \textbf{Response body:}
        \begin{codeblock}{JSON}
          \begin{minted}[breaklines, linenos, style=one-dark]{json}
{
  "id": "string($uuid)",
  "userId": "string($uuid)",
  "friendId": "string($uuid)",
  "status": "string",
  "createdAt": "string($date-time)",
  "updatedAt": "string($date-time)"
}
          \end{minted}
        \end{codeblock}

      \item \textbf{404 Not Found} - Nincs találat

        \textbf{Response body:}
        \begin{codeblock}{JSON}
          \begin{minted}[breaklines, linenos, style=one-dark]{json}
"User sent no friend requests."
          \end{minted}
        \end{codeblock}
      \end{itemize}

  \item[\httpPost{/api/friendship}] - Barát hozzáadása
  
    \vspace{0.5cm}
    \textbf{Request body:}
    \begin{codeblock}{JSON}
      \begin{minted}[breaklines, linenos, style=one-dark]{json}
{
  "userId": "string($uuid)",
  "friendId": "string($uuid)"
}
      \end{minted}
    \end{codeblock}

    \vspace{0.5cm}
    \textbf{Válaszok:}
    \begin{itemize}
      \item \textbf{201 Created} - Sikeres hozzáadás

        \textbf{Response body:}
        \begin{codeblock}{JSON}
          \begin{minted}[breaklines, linenos, style=one-dark]{json}
{
  "id": "string($uuid)",
  "createdAt": "string($date-time)",
  "email": "string($email)",
  "firstName": "string",
  "lastName": "string",
  "profilePictureBase64": "string",
  "longestStreak": Number
}
          \end{minted}
        \end{codeblock}

      \item \textbf{400 Bad Request} - Hibás vagy hiányos adatok

      \item \textbf{401 Unauthorized} - A felhasználó nincs bejelentkezve

      \item \textbf{404 Not Found} - Nincs találat

        \textbf{Response body:}
        \begin{codeblock}{JSON}
          \begin{minted}[breaklines, linenos, style=one-dark]{json}
"User not found."
          \end{minted}
        \end{codeblock}
    \end{itemize}

  \item[\httpPatch{/api/friendship}] - Barátság státuszának módosítása
  
    \vspace{0.5cm}
    \textbf{Request body:}
    \begin{codeblock}{JSON}
      \begin{minted}[breaklines, linenos, style=one-dark]{json}
{
  "userId": "string($uuid)",
  "friendId": "string($uuid)",
  "status": "string"
}
      \end{minted}
    \end{codeblock}

    \vspace{0.5cm}
    \textbf{Válaszok:}
    \begin{itemize}
      \item \textbf{200 OK} - Sikeres módosítás

        \textbf{Response body:}
        \begin{codeblock}{JSON}
          \begin{minted}[breaklines, linenos, style=one-dark]{json}
{
  "id": "string($uuid)",
  "userId": "string($uuid)",
  "friendId": "string($uuid)",
  "status": "string",
  "createdAt": "string($date-time)",
  "updatedAt": "string($date-time)"
}
          \end{minted}
        \end{codeblock}

      \item \textbf{400 Bad Request} - Hibás vagy hiányos adatok

      \item \textbf{401 Unauthorized} - A felhasználó nincs bejelentkezve

      \item \textbf{403 Forbidden} - Nem barátok a megadott felhasználóval
      
    \end{itemize}

  \item[\httpDelete{/api/friendship}] - Barát törlése
  
    \vspace{0.5cm}
    \textbf{Paraméterek:}
    \begin{itemize}
      \item \textbf{userId: string(\$uuid)} - A felhasználó azonosítója
      \item \textbf{friendId: string(\$uuid)} - A barát azonosítója
    \end{itemize}

    \vspace{0.5cm}
    \textbf{Válaszok:}
    \begin{itemize}
      \item \textbf{200 OK} - Sikeres törlés

        \textbf{Response body:}
        \begin{codeblock}{JSON}
          \begin{minted}[breaklines, linenos, style=one-dark]{json}
{
  "id": "string($uuid)",
  "createdAt": "string($date-time)",
  "email": "string($email)",
  "firstName": "string",
  "lastName": "string",
  "profilePictureBase64": "string",
  "longestStreak": Number
}
          \end{minted}
        \end{codeblock}

      \item \textbf{401 Unauthorized} - A felhasználó nincs bejelentkezve

      \item \textbf{400 Bad Request} - Hibás vagy hiányos adatok
      
      \item \textbf{403 Forbidden} - Nem barátok a megadott felhasználóval

      \item \textbf{404 Not Found} - Nincs találat

        \textbf{Response body:}
        \begin{codeblock}{JSON}
          \begin{minted}[breaklines, linenos, style=one-dark]{json}
"User not found."
          \end{minted}
        \end{codeblock}
    \end{itemize}
\end{description}

\subsection{Szokások}
\begin{description}
  \item[\httpGet{/api/habit}] - Szokások lekérdezése

    \vspace{0.5cm}
    A felhasználó szokásainak lekérdezése a felhasználó azonosítója alapján.

    \vspace{0.5cm}
    \textbf{Paraméterek:}
    \begin{itemize}
      \item \textbf{userId: string(\$uuid)} - A felhasználó azonosítója
      \item \textbf{pageNumber: int} - A lekérdezni kívánt oldal száma
      \item \textbf{pageSize: int} - A lekérdezni kívánt szokások száma
    \end{itemize}

    \vspace{0.5cm}
    \textbf{Válaszok:}
    \begin{itemize}
      \item \textbf{200 OK} - Sikeres lekérdezés

        \textbf{Response body:}
        \begin{codeblock}{JSON}
          \begin{minted}[breaklines, linenos, style=one-dark]{json}
{
  "id": "string($uuid)",
  "userId": "string($uuid)",
  "createdAt": "string($date-time)",
  "title": "string",
  "description": "string",
  "frequencyType": "string",
  "customFrequency": Number,
  "color": "string",
  "isPositive": bool,
  "streak": Number,
  "streakStart": "string($date-time)",
  "deleted": bool
}
          \end{minted}
        \end{codeblock}

      \item \textbf{401 Unauthorized} - A felhasználó nincs bejelentkezve

      \item \textbf{404 Not Found} - Nincs találat
      
        \textbf{Response body:}
        \begin{codeblock}{JSON}
          \begin{minted}[breaklines, linenos, style=one-dark]{json}
"User has no recorded habits."
          \end{minted}
        \end{codeblock}
    \end{itemize}

  \item[\httpGet{/api/habit/\{id\}}] - Szokás lekérdezése
  
  \vspace{0.5cm}
  Szokás lekérdezése azonosító alapján.

  \vspace{0.5cm}
  \textbf{Paraméterek:}
  \begin{itemize}
    \item \textbf{id: string(\$uuid)} - A szokás azonosítója
  \end{itemize}

  \vspace{0.5cm}
  \textbf{Válaszok:}
  \begin{itemize}
    \item \textbf{200 OK} - Sikeres lekérdezés

      \textbf{Response body:}
      \begin{codeblock}{JSON}
        \begin{minted}[breaklines, linenos, style=one-dark]{json}
{
  "id": "string($uuid)",
  "userId": "string($uuid)",
  "createdAt": "string($date-time)",
  "title": "string",
  "description": "string",
  "frequencyType": "string",
  "customFrequency": Number,
  "color": "string",
  "isPositive": bool,
  "streak": Number,
  "streakStart": "string($date-time)",
  "deleted": bool
}
        \end{minted}
      \end{codeblock}

    \item \textbf{401 Unauthorized} - A felhasználó nincs bejelentkezve

    \item \textbf{404 Not Found} - Nincs találat

      \textbf{Response body:}
      \begin{codeblock}{JSON}
        \begin{minted}[breaklines, linenos, style=one-dark]{json}
"Habit was not found."
        \end{minted}
      \end{codeblock}
  \end{itemize}

  \item[\httpPost{/api/habit}] - Szokás hozzáadása
  
    \vspace{0.5cm}
    \textbf{Request body:}
    \begin{codeblock}{JSON}
      \begin{minted}[breaklines, linenos, style=one-dark]{json}
{
  "title": "string",
  "userId": "string($uuid)",
  "description": "string",
  "frequencyType": "string",
    // DAILY, WEEKLY, MONTHLY, CUSTOM
  "customFrequency": Number,
    // Ha a frequencyType CUSTOM
  "color": "string",
    // HEX kód
  "isPositive": bool
}
      \end{minted}
    \end{codeblock}

    \vspace{0.5cm}
    \textbf{Válaszok:}
    \begin{itemize}
      \item \textbf{201 Created} - Sikeres hozzáadás

        \textbf{Response body:}
        \begin{codeblock}{JSON}
          \begin{minted}[breaklines, linenos, style=one-dark]{json}
{
  "id": "string($uuid)",
  "userId": "string($uuid)",
  "createdAt": "string($date-time)",
  "title": "string",
  "description": "string",
  "frequencyType": "string",
  "customFrequency": Number,
  "color": "string",
  "isPositive": bool,
  "streak": Number,
  "streakStart": "string($date-time)",
  "deleted": bool
}
          \end{minted}
        \end{codeblock}

      \item \textbf{400 Bad Request} - Hibás vagy hiányos adatok
      
      \item \textbf{401 Unauthorized} - A felhasználó nincs bejelentkezve

      \item \textbf{403 Forbidden} - A megadott azonosító nem egyezik a 
      \hfill \\ bejelentkezett felhasználóéval
      
        \textbf{Response body:}
        \begin{codeblock}{JSON}
          \begin{minted}[breaklines, linenos, style=one-dark]{json}
"You are not authorized to perform this action."
          \end{minted}
        \end{codeblock}
    \end{itemize}

  \item[\httpPatch{/api/habit/\{id\}}] - Szokás módosítása
  
    \vspace{0.5cm}
    \textbf{Paraméterek:}
    \begin{itemize}
      \item \textbf{id: string(\$uuid)} - A szokás azonosítója
    \end{itemize}

    \vspace{0.5cm}
    \textbf{Request body:}
    \begin{codeblock}{JSON}
      \begin{minted}[breaklines, linenos, style=one-dark]{json}
{
  "title": "string",
  "description": "string",
  "frequencyType": "string",
    // DAILY, WEEKLY, MONTHLY, CUSTOM
  "customFrequency": Number,
  "color": "string",
  "isPublic": bool
}
      \end{minted}
    \end{codeblock}

    \vspace{0.5cm}
    \textbf{Válaszok:}
    \begin{itemize}
      \item \textbf{200 OK} - Sikeres módosítás

        \textbf{Response body:}
        \begin{codeblock}{JSON}
          \begin{minted}[breaklines, linenos, style=one-dark]{json}
{
  "id": "string($uuid)",
  "userId": "string($uuid)",
  "createdAt": "string($date-time)",
  "title": "string",
  "description": "string",
  "frequencyType": "string",
  "customFrequency": Number,
  "color": "string",
  "isPositive": bool,
  "streak": Number,
  "streakStart": "string($date-time)",
  "deleted": bool
}
          \end{minted}
        \end{codeblock}
    
      \item \textbf{401 Unauthorized} - A felhasználó nincs bejelentkezve

      \item \textbf{404 Not Found} - Nincs találat

        \textbf{Response body:}
        \begin{codeblock}{JSON}
          \begin{minted}[breaklines, linenos, style=one-dark]{json}
"Habit was not found."
          \end{minted}
        \end{codeblock}
    \end{itemize}

  \item[\httpDelete{/api/habit/\{id\}}] - Szokás törlése
  
    \vspace{0.5cm}
    \textbf{Paraméterek:}
    \begin{itemize}
      \item \textbf{id: string(\$uuid)} - A szokás azonosítója
    \end{itemize}

    \vspace{0.5cm}
    \textbf{Válaszok:}
    \begin{itemize}
      \item \textbf{200 OK} - Sikeres törlés

        \textbf{Response body:}
        \begin{codeblock}{JSON}
          \begin{minted}[breaklines, linenos, style=one-dark]{json}
{
  "id": "string($uuid)",
  "userId": "string($uuid)",
  "createdAt": "string($date-time)",
  "title": "string",
  "description": "string",
  "frequencyType": "string",
  "customFrequency": Number,
  "color": "string",
  "isPositive": bool,
  "streak": Number,
  "streakStart": "string($date-time)",
  "deleted": bool
}
          \end{minted}
        \end{codeblock}

      \item \textbf{401 Unauthorized} - A felhasználó nincs bejelentkezve

      \item \textbf{404 Not Found} - Nincs találat

        \textbf{Response body:}
        \begin{codeblock}{JSON}
          \begin{minted}[breaklines, linenos, style=one-dark]{json}
"Habit not found."
          \end{minted}
        \end{codeblock}
    \end{itemize}
\end{description}

\subsection{Szokás napló}
\begin{description}
  \item[\httpGet{/api/habitLog}] - Szokás naplók lekérdezése
  
    \vspace{0.5cm}
    A szokás naplók lekérdezése a szokás azonosítója alapján.

    \vspace{0.5cm}
    \textbf{Paraméterek:}
    \begin{itemize}
      \item \textbf{habitId: string(\$uuid)} - A szokás azonosítója
      \item \textbf{pageNumber: int} - A lekérdezni kívánt oldal száma
      \item \textbf{pageSize: int} - A lekérdezni kívánt naplók száma
    \end{itemize}

    \vspace{0.5cm}
    \textbf{Válaszok:}
    \begin{itemize}
      \item \textbf{200 OK} - Sikeres lekérdezés

        \textbf{Response body:}
        \begin{codeblock}{JSON}
          \begin{minted}[breaklines, linenos, style=one-dark]{json}
{
  "id": "string($uuid)",
  "userId": "string($uuid)",
  "habitId": "string($uuid)",
  "dueDate": "string($date-time)",
  "completed": bool
  "comletedAt": "string($date-time)",
  "notes": "string",
  "isPublic": bool
}
          \end{minted}
        \end{codeblock}

      \item \textbf{401 Unauthorized} - A felhasználó nincs bejelentkezve

      \item \textbf{404 Not Found} - Nincs találat
      
        \textbf{Response body:}
        \begin{codeblock}{JSON}
          \begin{minted}[breaklines, linenos, style=one-dark]{json}
"No log was found for the provided habit."
          \end{minted}
        \end{codeblock}
    \end{itemize}

  \item[\httpGet{/api/habitLog/\{id\}}] - Szokás napló lekérdezése
  
    \vspace{0.5cm}
    Szokás napló lekérdezése azonosító alapján.

    \vspace{0.5cm}
    \textbf{Paraméterek:}
    \begin{itemize}
      \item \textbf{id: string(\$uuid)} - A szokás napló azonosítója
    \end{itemize}

    \vspace{0.5cm}
    \textbf{Válaszok:}
    \begin{itemize}
      \item \textbf{200 OK} - Sikeres lekérdezés

        \textbf{Response body:}
        \begin{codeblock}{JSON}
          \begin{minted}[breaklines, linenos, style=one-dark]{json}
{
  "id": "string($uuid)",
  "userId": "string($uuid)",
  "habitId": "string($uuid)",
  "dueDate": "string($date-time)",
  "completed": bool
  "comletedAt": "string($date-time)",
  "notes": "string",
  "isPublic": bool
}
          \end{minted}
        \end{codeblock}

      \item \textbf{401 Unauthorized} - A felhasználó nincs bejelentkezve

      \item \textbf{404 Not Found} - Nincs találat

        \textbf{Response body:}
        \begin{codeblock}{JSON}
          \begin{minted}[breaklines, linenos, style=one-dark]{json}
"Habit log was not found."
          \end{minted}
        \end{codeblock}
    \end{itemize}

  \item[\httpGet{/api/habitlog/\{dueDate\}}] 
  \hfill \\ - Szokás napló lekérdezése határidő alapján
  
    \vspace{0.5cm}
    \textbf{Paraméterek:}
    \begin{itemize}
      \item \textbf{userId: string(\$uuid)} - A felhasználó azonosítója
      \item \textbf{dueDate: string(\$date)} - A szokás napló dátuma
    \end{itemize}

    \vspace{0.5cm}
    \textbf{Válaszok:}
    \begin{itemize}
      \item \textbf{200 OK} - Sikeres lekérdezés

        \textbf{Response body:}
        \begin{codeblock}{JSON}
          \begin{minted}[breaklines, linenos, style=one-dark]{json}
{
  "id": "string($uuid)",
  "userId": "string($uuid)",
  "habitId": "string($uuid)",
  "dueDate": "string($date-time)",
  "completed": bool
  "comletedAt": "string($date-time)",
  "notes": "string",
  "isPublic": bool
}
          \end{minted}
        \end{codeblock}

      \item \textbf{401 Unauthorized} - A felhasználó nincs bejelentkezve

      \item \textbf{404 Not Found} - Nincs találat
      
        \textbf{Response body:}
        \begin{codeblock}{JSON}
          \begin{minted}[breaklines, linenos, style=one-dark]{json}
"No habit log was found with the provided due date."
          \end{minted}
        \end{codeblock}
    \end{itemize}

  \item[\httpPatch{/api/habitLog/\{id\}}] - Szokás napló módosítása
  
    \vspace{0.5cm}
    \textbf{Paraméterek:}
    \begin{itemize}
      \item \textbf{id: string(\$uuid)} - A szokás napló azonosítója
    \end{itemize}

    \vspace{0.5cm}
    \textbf{Request body:}
    \begin{codeblock}{JSON}
      \begin{minted}[breaklines, linenos, style=one-dark]{json}
{
  "completed": bool,
  "completedAt": "string($date-time)",
  "notes": "string",
  "isPublic": bool
}
      \end{minted}
    \end{codeblock}

    \vspace{0.5cm}
    \textbf{Válaszok:}
    \begin{itemize}
      \item \textbf{200 OK} - Sikeres módosítás

        \textbf{Response body:}
        \begin{codeblock}{JSON}
          \begin{minted}[breaklines, linenos, style=one-dark]{json}
{
  "id": "string($uuid)",
  "userId": "string($uuid)",
  "habitId": "string($uuid)",
  "dueDate": "string($date-time)",
  "completed": bool
  "comletedDate": "string($date-time)",
  "notes": "string",
  "isPublic": bool
}
          \end{minted}
        \end{codeblock}

      \item \textbf{401 Unauthorized} - A felhasználó nincs bejelentkezve

      \item \textbf{404 Not Found} - Nincs találat

        \textbf{Response body:}
        \begin{codeblock}{JSON}
          \begin{minted}[breaklines, linenos, style=one-dark]{json} 
"Habit log was not found."
          \end{minted}
        \end{codeblock}
    \end{itemize}
\end{description}

\subsection{Feed}
\begin{description}
  \item[\httpGet{/api/feed}] - Feed lekérdezése
  
    \vspace{0.5cm}
    Más felhasználók által publikusként megjelölt szokás naplók lekérdezése.

    \vspace{0.5cm}
    \textbf{Paraméterek:}
    \begin{itemize}
      \item \textbf{pageNumber: int} - A lekérdezni kívánt oldal száma
      \item \textbf{pageSize: int} - A lekérdezni kívánt bejegyzések száma
    \end{itemize}

    \vspace{0.5cm}
    \textbf{Válaszok:}
    \begin{itemize}
      \item \textbf{200 OK} - Sikeres lekérdezés

        \textbf{Response body:}
        \begin{codeblock}{JSON}
          \begin{minted}[breaklines, linenos, style=one-dark]{json}
{
  "id": "string($uuid)",
  "userId": "string($uuid)",
  "habitId": "string($uuid)",
  "dueDate": "string($date-time)",
  "completed": bool,
  "completedAt": "string($date-time)",
  "notes": "string",
  "isPublic": bool
}
          \end{minted}
        \end{codeblock}

      \item \textbf{404 Not Found} - Nincs találat
      
        \textbf{Response body:}
        \begin{codeblock}{JSON}
          \begin{minted}[breaklines, linenos, style=one-dark]{json}
"Could not update feed."
          \end{minted}
        \end{codeblock}
    \end{itemize}
\end{description}

\subsection{Kihívások}
\begin{description}
  \item[\httpGet{/api/challenge/\{userId\}/challenge-invites}] - Kihívási felkérések lekérdezése
  
    \vspace{0.5cm}
    A felhasználó által kapott kihívási felkérések lekérdezése.

    \vspace{0.5cm}
    \textbf{Paraméterek:}
    \begin{itemize}
      \item \textbf{userId: string(\$uuid)} - A felhasználó azonosítója
    \end{itemize}

    \vspace{0.5cm}
    \textbf{Válaszok:}
    \begin{itemize}
      \item \textbf{200 OK} - Sikeres lekérdezés

        \textbf{Response body:}
        \begin{codeblock}{JSON}
          \begin{minted}[breaklines, linenos, style=one-dark]{json}
{
  "id": "string($uuid)",
  "userId": "string($uuid)",
  "friendId": "string($uuid)",
  "habit": 
  {
    "id": "string($uuid)",
    "userId": "string($uuid)",
    "challengedFriends": [
      "string($uuid)"
    ],
    "createdAt": "string($date-time)",
    "title": "string",
    "description": "string",
    "frequencyType": "string",
    "customFrequency": Number,
    "color": "string",
    "isPositive": bool,
    "streak": Number,
    "streakStart": "string($date-time)",
    "deleted": bool
  },
  "status": "string",
  "createdAt": "string($date-time)",
  "updatedAt": "string($date-time)"
}
          \end{minted}
        \end{codeblock}

      \item \textbf{401 Unauthorized} - A felhasználó nincs bejelentkezve

      \item \textbf{404 Not Found} - Nincs találat

        \textbf{Response body:}
        \begin{codeblock}{JSON}
          \begin{minted}[breaklines, linenos, style=one-dark]{json}
"User has no challenge invites."
          \end{minted}
        \end{codeblock}

      \end{itemize}

  \item[\httpPost{/api/challenge}] - Barát kihívása
  
    \vspace{0.5cm}
    \textbf{Request body:}
    \begin{codeblock}{JSON}
      \begin{minted}[breaklines, linenos, style=one-dark]{json}
{
  "userId": "string($uuid)",
  "friendId": "string($uuid)",
  "habit": 
  {
    "id": "string($uuid)",
    "userId": "string($uuid)",
    "challengedFriends": [
      "string($uuid)"
    ],
    "createdAt": "string($date-time)",
    "title": "string",
    "description": "string",
    "frequencyType": "string",
    "customFrequency": Number,
    "color": "string",
    "isPositive": bool,
    "streak": Number,
    "streakStart": "string($date-time)",
    "deleted": bool
  }
}
      \end{minted}
    \end{codeblock}

    \vspace{0.5cm}
    \textbf{Válaszok:}
    \begin{itemize}
      \item \textbf{201 Created} - Sikeres küldés

        \textbf{Response body:}
        \begin{codeblock}{JSON}
          \begin{minted}[breaklines, linenos, style=one-dark]{json}
{
  "id": "string($uuid)",
  "userId": "string($uuid)",
  "friendId": "string($uuid)",
  "habit": 
  {
    "id": "string($uuid)",
    "userId": "string($uuid)",
    "challengedFriends": [
      "string($uuid)"
    ],
    "createdAt": "string($date-time)",
    "title": "string",
    "description": "string",
    "frequencyType": "string",
    "customFrequency": Number,
    "color": "string",
    "isPositive": bool,
    "streak": Number,
    "streakStart": "string($date-time)",
    "deleted": bool
  },
  "status": "string",
  "createdAt": "string($date-time)",
  "updatedAt": "string($date-time)"
}
          \end{minted}
        \end{codeblock}

      \item \textbf{400 Bad Request} - Hibás vagy hiányos adatok

      \item \textbf{401 Unauthorized} - A felhasználó nincs bejelentkezve

      \item \textbf{404 Not Found} - Nincs találat

        \textbf{Response body:}
        \begin{codeblock}{JSON}
          \begin{minted}[breaklines, linenos, style=one-dark]{json}
"Habit not found"
          \end{minted}
        \end{codeblock}
    \end{itemize}

  \item[\httpPatch{/api/challenge/\{id\}}] - Kihívás módosítása

    \vspace{0.5cm}
    \textbf{Request body:}
    \begin{codeblock}{JSON}
      \begin{minted}[breaklines, linenos, style=one-dark]{json}
{
  "userId": "string($uuid)",
  "friendId": "string($uuid)",
  "habit": 
  {
    "id": "string($uuid)",
    "userId": "string($uuid)",
    "challengedFriends": [
      "string($uuid)"
    ],
    "createdAt": "string($date-time)",
    "title": "string",
    "description": "string",
    "frequencyType": "string",
    "customFrequency": Number,
    "color": "string",
    "isPositive": bool,
    "streak": Number,
    "streakStart": "string($date-time)",
    "deleted": bool
  },
  "status": "string"
}
      \end{minted}
    \end{codeblock}

    \vspace{0.5cm}
    \textbf{Válaszok:}
    \begin{itemize}
      \item \textbf{200 OK} - Sikeres módosítás

        \textbf{Response body:}
        \begin{codeblock}{JSON}
          \begin{minted}[breaklines, linenos, style=one-dark]{json}
{
  "id": "string($uuid)",
  "userId": "string($uuid)",
  "friendId": "string($uuid)",
  "habit": 
  {
    "id": "string($uuid)",
    "userId": "string($uuid)",
    "challengedFriends": [
      "string($uuid)"
    ],
    "createdAt": "string($date-time)",
    "title": "string",
    "description": "string",
    "frequencyType": "string",
    "customFrequency": Number,
    "color": "string",
    "isPositive": bool,
    "streak": Number,
    "streakStart": "string($date-time)",
    "deleted": bool
  },
  "status": "string",
  "createdAt": "string($date-time)",
  "updatedAt": "string($date-time)"
}
          \end{minted}
        \end{codeblock}

      \item \textbf{401 Unauthorized} - A felhasználó nincs bejelentkezve

      \item \textbf{400 Bad Request} - Hibás vagy hiányos adatok

    \end{itemize}

  \item[\httpDelete{/api/challenge/\{id\}}] - Kihívás törlése
  
    \vspace{0.5cm}
    \textbf{Paraméterek:}
    \begin{itemize}
      \item \textbf{habitId: string(\$uuid)} - A szokás azonosítója
    \end{itemize}

    \vspace{0.5cm}
    \textbf{Válaszok:}
    \begin{itemize}
      \item \textbf{200 OK} - Sikeres törlés

        \textbf{Response body:}
        \begin{codeblock}{JSON}
          \begin{minted}[breaklines, linenos, style=one-dark]{json}
{
  "id": "string($uuid)",
  "userId": "string($uuid)",
  "friendId": "string($uuid)",
  "habit": 
  {
    "id": "string($uuid)",
    "userId": "string($uuid)",
    "challengedFriends": [
      "string($uuid)"
    ],
    "createdAt": "string($date-time)",
    "title": "string",
    "description": "string",
    "frequencyType": "string",
    "customFrequency": Number,
    "color": "string",
    "isPositive": bool,
    "streak": Number,
    "streakStart": "string($date-time)",
    "deleted": bool
  },
  "status": "string",
  "createdAt": "string($date-time)",
  "updatedAt": "string($date-time)"
}
          \end{minted}
        \end{codeblock}

      \item \textbf{401 Unauthorized} - A felhasználó nincs bejelentkezve

      \item \textbf{404 Not Found} - Nincs találat

        \textbf{Response body:}
        \begin{codeblock}{JSON}
          \begin{minted}[breaklines, linenos, style=one-dark]{json}
"Failed to remove challenge."
          \end{minted}
        \end{codeblock}
    \end{itemize}

\end{description}

\chapter{Tesztelés}

\end{document}
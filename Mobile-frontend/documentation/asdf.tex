\documentclass[12pt,a4paper]{report}

% Packages
\usepackage[utf8]{inputenc}
\usepackage{graphicx}
\usepackage{hyperref}
\usepackage{listings}
\usepackage{xcolor}
\usepackage{caption}
\usepackage{float}

% Code listing style
\definecolor{codegray}{rgb}{0.5,0.5,0.5}
\definecolor{codepurple}{rgb}{0.58,0,0.82}
\definecolor{backcolour}{rgb}{0.95,0.95,0.92}
 
\lstdefinestyle{mystyle}{
  backgroundcolor=\color{backcolour},   
  commentstyle=\color{codegray},
  keywordstyle=\color{blue},
  numberstyle=\tiny\color{codegray},
  stringstyle=\color{codepurple},
  basicstyle=\ttfamily\footnotesize,
  breakatwhitespace=false,         
  breaklines=true,                 
  captionpos=b,                    
  keepspaces=true,                 
  numbers=left,                    
  numbersep=5pt,                  
  showspaces=false,                
  showstringspaces=false,
  showtabs=false,                  
  tabsize=2
}
\lstset{style=mystyle}

% Document information
\title{\Huge{\textbf{Elevate Alkalmazás}}\\\Large{Műszaki Dokumentáció}}
\author{Elevate Projekt Csapat}
\date{\today}

\begin{document}

\maketitle
\tableofcontents
\newpage

\chapter{Bevezetés}

\section{Áttekintés}
Az Elevate egy Ionic és Angular technológiákkal készült mobilalkalmazás, amelyet arra terveztek, hogy segítsen a felhasználóknak pozitív szokások követésében és fenntartásában. Az alkalmazás lehetővé teszi a felhasználók számára, hogy létrehozzanak, nyomon kövessenek és megosszanak szokásokat barátaikkal, elősegítve ezzel az elszámoltathatóságot és a társadalmi támogatást az egészséges rutinok kialakításában.

\section{Cél}
Az alkalmazás célja, hogy intuitív és lebilincselő platformot biztosítson a felhasználók számára, ahol:
\begin{itemize}
    \item Napi, heti, havi vagy egyéni szokásokat hozhatnak létre és követhetnek
    \item Figyelemmel kísérhetik előrehaladásukat és fenntarthatják sorozataikat
    \item Kihívhatják barátaikat pozitív szokások elsajátítására
    \item Vizualizálhatják a szokások teljesítését naptárnézetben
    \item Értesítéseket és emlékeztetőket kaphatnak
\end{itemize}

\chapter{Architektúra}

\section{Technológiai Stack}
\begin{itemize}
    \item Frontend: Angular 17 Ionic keretrendszerrel
    \item UI Komponensek: Ionic UI komponensek
    \item Állapotkezelés: Angular szolgáltatások és observable-ök
    \item HTTP Kommunikáció: Angular HttpClient
    \item Hitelesítés: JWT token-alapú hitelesítés
\end{itemize}

\section{Projekt Struktúra}
A projekt a standard Angular komponens-alapú architektúrát követi, szolgáltatásokkal az üzleti logikához:

\begin{lstlisting}[language=bash]
Mobile-frontend/
|-- src/
|   |-- app/
|   |   |-- components/
|   |   |   |-- task-card/
|   |   |   |-- friend/
|   |   |   |-- header/
|   |   |-- pages/
|   |   |   |-- calendar/
|   |   |   |-- create-habit/
|   |   |   |-- login-page/
|   |   |-- services/
|   |   |   |-- habit.service.ts
|   |   |   |-- friendship.service.ts
|   |   |   |-- user.service.ts
|   |   |-- .models/
|   |   |   |-- Habit.model.ts
|   |   |   |-- HabitLog.model.ts
|   |   |   |-- User.model.ts
\end{lstlisting}

\chapter{Alap Funkciók}

\section{Szokáskezelés}
\subsection{Szokások Létrehozása}
A felhasználók a következő tulajdonságokkal hozhatnak létre szokásokat:
\begin{itemize}
    \item Cím és leírás
    \item Gyakoriság (napi, heti, havi vagy egyéni)
    \item Egyéni gyakoriság esetén, a hét meghatározott napjai
    \item Szín a vizuális azonosításhoz
    \item Pozitív vagy negatív szokások követése
\end{itemize}

\subsection{Egyéni Gyakoriság Implementációja}
Az alkalmazás bináris reprezentációt használ az egyéni gyakoriságú szokások követésére:

\begin{lstlisting}[language=typescript]
// Bináris reprezentáció az egyéni gyakorisághoz
// Minden bit a hét egy napját jelöli (hétfőtől vasárnapig)
// 1 = kiválasztott nap, 0 = nem kiválasztott nap
// Példa: 65 (decimális) = 1000001 (bináris) = hétfő és vasárnap

toggleDay(habit: Habit, dayIndex: number) {
  if (habit.customFrequency == null) {
    habit.customFrequency = 0; // Initialize to 0 if null
  }

  let binaryString = habit.customFrequency.toString(2).padStart(7, '0');
  let binaryArray = binaryString.split('');
  binaryArray[dayIndex] = binaryArray[dayIndex] === '1' ? '0' : '1';
  habit.customFrequency = parseInt(binaryArray.join(''), 2);
}

isSelectedDay(customFrequency: number, dayIndex: number): boolean {
  if (customFrequency != null) {
    const binaryString = customFrequency.toString(2).padStart(7, '0');
    return binaryString.charAt(dayIndex) === '1';
  }
  return false;
}
\end{lstlisting}

\section{Szokások Követése}
\subsection{Naptár Nézet}
A naptár oldal vizuális reprezentációt nyújt a szokások teljesítéséről:
\begin{itemize}
    \item Navigálás a napok között
    \item Adott napokra esedékes szokások megtekintése
    \item Szokások teljesítettként jelölése
    \item Sorozat információk megtekintése
\end{itemize}

\subsection{Szokás Teljesítése}
Amikor egy szokást teljesítenek:
\begin{itemize}
    \item A sorozat számláló frissül
    \item A teljesítés rögzítésre kerül a szokásnaplóban
    \item A felhasználók választhatnak, hogy a teljesítések nyilvánosak vagy privátak legyenek
\end{itemize}

\section{Közösségi Funkciók}
\subsection{Barát Rendszer}
\begin{itemize}
    \item Barátkérések küldése és elfogadása
    \item Barátok nyilvános szokásteljesítéseinek megtekintése
    \item Barátok kihívása szokások elsajátítására
\end{itemize}

\subsection{Kihívások}
A felhasználók kihívhatják barátaikat szokásaik átvételére:
\begin{itemize}
    \item Szokás kiválasztása megosztásra
    \item Kihívandó barátok kiválasztása
    \item Nyomon követés, hogy ki fogadta el a kihívásokat
\end{itemize}

\chapter{Kulcsfontosságú Komponensek}

\section{Feladat Kártya Komponens}
A Feladat Kártya komponens egy központi UI elem, amely megjeleníti a szokásokat és azok részleteit:

\begin{lstlisting}[language=html]
<ion-accordion-group expand="inset">
  @for(habit of habits ; track $index) {
  <ion-accordion [disabled]="isHabitCompleted(habit.id) || Disabled">
    <ion-item slot="header" color="light" class="habit-item">
      <div class="color" [style.backgroundColor]="habit.color"></div>
      <ion-input [readonly]="true" type="text" [value]="habit.title"></ion-input>
      <div class="time" slot="end">
        <ion-label>{{habit.streak}}</ion-label>
        <ion-icon name="flame-outline"></ion-icon>
      </div>
      <!-- Other elements -->
    </ion-item>
    <!-- Accordion content -->
  </ion-accordion>
  }
</ion-accordion-group>
\end{lstlisting}

\section{Naptár Oldal}
A Naptár oldal lehetővé teszi a felhasználók számára a dátumok közötti navigálást és a szokások megtekintését:

\begin{lstlisting}[language=typescript]
previousDay() {
  this.date.setDate(this.date.getDate() - 1);
  this.updateDateDisplay();
  this.cdr.detectChanges();
}

nextDay() {
  this.date.setDate(this.date.getDate() + 1);
  this.updateDateDisplay();
}

updateDateDisplay() {
  this.weekday = this.date.toLocaleDateString('en-US', { weekday: 'short' });
  this.dayAndMonth = this.date.toLocaleDateString('en-US', 
                            { day: '2-digit', month: 'short' });
  this.datestring = this.date.toISOString();
}
\end{lstlisting}

\section{Szokás Létrehozása Oldal}
A Szokás Létrehozása oldal felületet biztosít a felhasználók számára új szokások definiálásához:
\begin{itemize}
    \item Beviteli mezők a szokás tulajdonságaihoz
    \item Egyéni gyakoriság kiválasztási felület
    \item Színválasztó a szokás vizualizációjához
\end{itemize}

\chapter{Adatmodellek}

\section{Szokás Modell}
\begin{lstlisting}[language=typescript]
export interface Habit {
    id: string;
    created_at: Date;
    title: string;
    description?: string;
    frequencyType: Frequency;
    customFrequency: number;
    color: string;
    is_positive: boolean;
    streak: number;
    streak_start?: Date;
    challengedFriends?: string[];
    userId?: string;
}

export enum Frequency {
    Daily = 'daily',
    Weekly = 'weekly',
    Monthly = 'monthly',
    Custom = 'custom'
}
\end{lstlisting}

\section{Szokásnapló Modell}
\begin{lstlisting}[language=typescript]
export interface HabitLog {
    id: string;
    userId: string;
    habitId: string;
    dueDate: Date;
    completed: boolean;
    completedAt?: Date | null;
    notes?: string | null;
    isPublic: boolean;
}
\end{lstlisting}

\chapter{Alap Szolgáltatások}

\section{Szokás Szolgáltatás}
A Szokás Szolgáltatás kezeli az összes szokással kapcsolatos API műveletet:

\begin{lstlisting}[language=typescript]
@Injectable({
  providedIn: 'root'
})
export class HabitService {
  private http = inject(HttpClient);
  private apiUrl = 'https://elevate.koyeb.app/api/habit';

  getHabits(userId: string, pageNumber: number, pageSize: number): Observable<Habit[]> {
    // Implementáció
  }
  
  getHabitByID(habitId: string): Observable<any> {
    // Implementáció
  }
  
  getTodaysHabitlogs(date: string): Observable<any> {
    // Implementáció
  }

  createHabit(habitData: any): Observable<Habit> {
    // Implementáció
  }

  editHabit(Editedhabit: Habit) {
    // Implementáció
  }

  deleteHabit(id: string) {
    // Implementáció
  }

  completeHabit(habitLogId: string, Ispublic: boolean): Observable<any> {
    // Implementáció
  }

  sendChallenge(habit: Habit, friendId: string): Observable<any> {
    // Implementáció
  }
}
\end{lstlisting}

\chapter{Felhasználói Élmény}

\section{Navigáció}
Az alkalmazás intuitív navigációt biztosít:
\begin{itemize}
    \item Láblécfülek a fő szekciókhoz
    \item FAB gombok a gyors műveletekhez
    \item Csúsztatás gesztusok a dátum navigáláshoz
    \item Modális párbeszédablakok a gyors adatbevitelhez
\end{itemize}

\section{Vizuális Visszajelzés}
\begin{itemize}
    \item Színkódolt szokások a vizuális azonosításhoz
    \item Sorozat számlálók láng ikonokkal
    \item Trófea jelvények a kihívott szokásokhoz
    \item Toast üzenetek a műveletek megerősítéséhez
\end{itemize}

\section{Reszponzív Dizájn}
Az alkalmazást úgy tervezték, hogy különböző képernyőméreteken és tájolásokban működjön:
\begin{itemize}
    \item Rugalmas rács elrendezések
    \item Reszponzív UI komponensek
    \item Adaptív tipográfia
    \item Érintésbarát interakciós célpontok
\end{itemize}

\chapter{Tesztelés}

\section{Egységtesztelés}
Az alkalmazás egységteszteket tartalmaz a következők használatával:
\begin{itemize}
    \item Jasmine tesztelési keretrendszer
    \item Karma teszt futtató
    \item Angular tesztelési segédprogramok
\end{itemize}

\begin{lstlisting}[language=typescript]
describe('HabitService', () => {
  let service: HabitService;

  beforeEach(() => {
    TestBed.configureTestingModule({});
    service = TestBed.inject(HabitService);
  });

  it('should be created', () => {
    expect(service).toBeTruthy();
  });
});
\end{lstlisting}

\section{Böngésző Kompatibilitás}
Az alkalmazást úgy tervezték, hogy támogassa a következő böngészőket:
\begin{itemize}
    \item Chrome ≥79
    \item ChromeAndroid ≥79
    \item Firefox ≥70
    \item Edge ≥79
    \item Safari ≥14
    \item iOS ≥14
\end{itemize}

\chapter{Jövőbeli Fejlesztések}

\section{Tervezett Funkciók}
A projekt munkaterve alapján:
\begin{itemize}
    \item Továbbfejlesztett push értesítések
    \item Sötét/világos téma opciók
    \item Továbbfejlesztett vizualizáció a naptár nézetben
    \item További közösségi funkciók
    \item Átfogóbb tesztelés
\end{itemize}

\section{Push Értesítési Stratégia}
Tervezett értesítési típusok:
\begin{itemize}
    \item Napi emlékeztetők a szokások teljesítésére
    \item Sorozat mérföldkövek ünneplése
    \item Ösztönző üzenetek a negatív szokások elkerülésére
    \item Kihívás meghívók és frissítések
\end{itemize}

\chapter{Összegzés}
Az Elevate átfogó megoldást nyújt a szokások követésére közösségi funkciókkal, segítve a felhasználókat pozitív rutinok kialakításában elszámoltathatóság, vizualizáció és baráti kihívások révén. Az alkalmazás moduláris architektúrája lehetővé teszi a könnyű karbantartást és a funkciók bővítését.

\end{document}